% !TEX program = xelatex
\documentclass[12pt]{resume}
\usepackage{fontawesome7}
\usepackage{enumitem}
\usepackage{hyperref}
\usepackage{geometry}
\usepackage{titlesec}
\usepackage{hanging}

\begin{document}
\pagenumbering{gobble} % suppress displaying page number

\name{Chentao Fan}

\basicInfo{
  \email{cxf373@case.edu} \textperiodcentered\
  \phone{(216)303-2715} \textperiodcentered\
  \github[Aphcity]{https://github.com/Aphcity} \textperiodcentered\
  \linkedin[Linkedin]{https://www.linkedin.com/in/chentao-fan-9192a5324/} \textperiodcentered\
  \homepage[Homepage]{https://aphcity.top} \textperiodcentered\
  \orcid[OrcID]{0009-0006-9853-9005} \
}

\section{\texorpdfstring{\textbf{\faGraduationCap}\ Education}{Education}}
\datedsubsection{\textbf{Case Western Reserve University}, Cleveland, U.S.A.}{2024 -- Present}
\textit{Master of Science} in Computer Science (CS), expected May 2027
\datedsubsection{\textbf{China University of Mining \& Technology(Beijing)}, Beijing, China}{2018 -- 2022}
\textit{Bachelor of Engineering} in Electronics Engineering and Automation (EE)

\section{\texorpdfstring{\textbf{\faCity}\ Internships}{Internships}}
\datedsubsection{\textbf{Shanxi Sensor \& Control Electronics Technology Co., Ltd}, Shanxi, China}{Jun. 2024 -- Aug. 2024}
\role{C, keil, ARM, Linux}{Software Engineer}
Brief introduction:
Using STC8H8K64U MCU and OneNet Open IoT to control the valve.
\begin{itemize}
  \item Implemented angle control and NFC-controlled function of valves.
  \item Optimized the command protocol on both client and platform sides.
  \item Developed the PWM-controlled buzzer module with adjustable frequency and duty cycle.
  \item Implemented an ADC module-controlled motor with temperature and angle sensors.
  \item Composed complete reference documents for this project.
\end{itemize}

\datedsubsection{\textbf{Beijing Glory PKPM Technology Co., Ltd}, Beijing, China}{Jun. 2022 -- Jul. 2022}
\role{Python}{Software Engineer}
Brief introduction:
Using STC8H8K64U MCU and OneNet Open IoT to control the valve.
\begin{itemize}
  \item Learned to set up BIM Base and Python for a Visual Studio Code development environment.
  \item Adopted the “pyp3d” library for parametric modeling.
  \item Conducted parametric modeling experiments on various 3D shapes.
  \item Imported the `.py` file into the BIM Base Kit.
  \item Packed the model into a BFA file to encrypt model data and prevent script leakage.
\end{itemize}

% Publications

\section{\texorpdfstring{\textbf{\faBuildingColumns}\ Publications}{Publications}}

\datedsubsection{\textbf{AQIoT: Partnering with Community Organizations to Provide Fine-Grained Air Quality Data}}{Nov 3rd, 2025}
\role{\href{https://doi.org/10.1145/3764924.3770896}{10.1145/3764924.3770896}}{Conference paper}
Contributors: Brian Krupp; Jonathon Fagert; Chentao Fan; Isabelle Zhang; Dorian Hawkins.


\section{\texorpdfstring{\textbf{\faListCheck}\ Projects}{Projects}}

\datedsubsection{\textbf{AQIoT @ CASLab CWRU}}{Oct. 2024 -- Present}
\role{Python, Shell Script, Kotlin, RESTful API}{Research Assistant}
Brief introduction: A Raspberry Pi-based air quality monitoring system
\begin{itemize}
  \item Develop a shell script to customize RasPi image with the necessary software and libraries and certain configurations.
  \item Utilize Python and Adafruit AQ sensor libraries to read air quality data on Raspberry Pi.
  \item Implement a Linux systemd service to run the Python script in the background.
  \item Add documentation and instructions and GitHub rules for the project.
  \item Develop an Android app to visualize the data through RESTful API
\end{itemize}

\datedsubsection{\textbf{Dual Channel Watermarking Using Machine Learning}}{Jan. 2025 -- May. 2025}
\role{Python, TensorFlow, PyTorch, Kotlin}{Individual Project}
Brief introduction: A dual-channel watermarking system using machine learning techniques.
\begin{itemize}
  \item Implemented a deep learning model using TensorFlow and PyTorch for watermark detection and extraction.
  \item Developed a Kotlin-based Android app for real-time watermarking and dewatermarking of images.
  \item Conducted experiments to evaluate the robustness of the watermarking technique against various attacks.
  \item Documented the project and created a user manual for the Android app.
\end{itemize}

\datedsubsection{\textbf{Merriam-Webster Dictionary Wrapper}}{Mar. 2023 -- Sep. 2023}
\role{Python}{Individual Project}
Brief introduction: Offline Merriam-Webster dictionary -- scraped and parsed with Python (requests, lxml), saved as local HTML and exported to .xls, then packaged with MDX Builder into an .mdx file for MDict.

\datedsubsection{\textbf{Personal Blog}}{Feb. 2022 -- Present}
\role{Markdown, LaTeX, HTML, Linux, CI/CD}{Individual Project}
Brief introduction: Now using Hugo (Hexo before) as the framework, hosted a website on GitHub Pagesand Vercel.
\begin{itemize}
  \item Enhanced module appearance via CSS files.
  \item Employed GitHub Actions to monitor remote branch changes and update the website.
  \item Utilized DNSpod to perform domestic and international line traffic controling (Vercel for CN GEOIP and GitHub Pages for others) and domain binding.
  \item Used Twikoo as the comment module and deployed the database file containing reviewer information and comment-related content into MongoDB through Vercel.
  \item Utilized the sm.ms image hosting service API to automatically upload reviewer images to the image bed.
  \item Employed Algolia as the index module to fulfill title matching and full-text search functions.
  \item Used \textit{Statically} to compress website static resources to increase access speed.
\end{itemize}

\datedsubsection{\textbf{Cloud Server Maintenance and Applications}}{Nov. 2022 -- Nov. 2023}
\role{Linux, Docker}{Individual Project}
Brief introduction: Configured Docker environment in Ubuntu to set up Minecraft servers suitable across versions by separating specific JRE environment versions; set up many game servers and deployed a music bot for KOOK (IM software like Discord) via open source docker image

\section{\texorpdfstring{\textbf{\faUserGroup}\ Group Experience}{Group Experience}}
\datedsubsection{\textbf{Computing and Society Lab@CWRU}}{ Oct. 2024 -- Present}
\role{Research Assistant}{}
Brief introduction: Researching on project focusing on providing air quality sensing in urban environments.

\datedsubsection{\textbf{Student Union of CUMTB}}{Sep. 2018 -- Aug. 2020}
\role{Administrative Officer}{Department Leader}
Brief introduction: Served as an officer (for the first year) and administrator of the Science and Technology Department.

\section{\texorpdfstring{\textbf{\faGears}\ Skills}{Skills}}
\begin{itemize}[parsep=0.5ex]
  \item Programming Languages: C == C++ > Python > Kotlin = Java
  \item Platforms: Windows, Linux, Android
  \item Other Skills: Photoshop, Premiere, After Effects, Piano, Soccer, Painting
  \item Languages: English - Fluent, Mandarin - Native speaker
\end{itemize}

% \section{\texorpdfstring{\textbf{\faCircleInfo}}\ Miscellaneous}{Miscellaneous}}
% \begin{itemize}[parsep=0.5ex]
%   \item Blog: \url{https://aphcity.top}
%   \item GitHub: \url{https://github.com/Aphcity}
%   
% \end{itemize}

%% References (if needed)
%\newpage
%\bibliographystyle{IEEETran}
%\bibliography{mycite}
\end{document}
